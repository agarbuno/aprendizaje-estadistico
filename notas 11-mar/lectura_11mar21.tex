\documentclass[11pt,reqno,twoside]{article}
%>>>>>>> RENAME CURRENT FILE TO MATCH LECTURE NUMBER
% E.g., "lecture_01.tex"

%>>>>>>> DO NOT EDIT MACRO FILE

% >>>> DO NOT EDIT THIS FILE
% If you *must* add or change a macro, please email fkoh@caltech.edu

%=================================================
% Basics
%=================================================

\usepackage{fixltx2e} % Makes \( \) equation style robust, among other
                      % things. Must be the first package.


% Makes ligatured fonts searchable and copyable in pdf readers
\usepackage{cmap} % Load before fontenc
\usepackage[spanish]{babel}

% Always include these font encodings in your document
% unless you have a very good reason.
\usepackage[T1]{fontenc}
\usepackage[utf8]{inputenc}

\usepackage{verbatim}

%=====================================
% Look & feel
%=====================================

% Allows for manual space setting
\usepackage{setspace}

%=============
% Fonts
%=============

\usepackage{lmodern} % Improved version of computer modern
\usepackage[scale=0.88]{tgheros} % Helvetica clone for sans serif font


\newcommand\hmmax{2} % Default is 3.
\newcommand\bmmax{2} % Default is 4.

\usepackage{bm} % boldmath must be called after the package
\providecommand{\mathbold}[1]{\bm{#1}}
\usepackage{dsfont}
%=============
% AMS Packages and fonts
%=============
\usepackage{amsmath,amsbsy,amsgen,amscd,amsthm,amsfonts,amssymb}

%=============
% Margins and paper size
%=============
\usepackage[centering,top=1.5in,bottom=1.2in,left=1.4in,right=1.4in]{geometry}

%=============
% Title setup
%=============
\usepackage{titling}
%\usepackage{nopageno}
\setlength{\droptitle}{-7.5em}

\pretitle{\noindent\rule{0.85\linewidth}{0.2mm}\par%
  \begin{raggedright}\LARGE\sffamily}
\posttitle{\par\end{raggedright}%
\noindent%
\rule{0.85\linewidth}{0.5mm}\par}

\preauthor{\noindent\vspace{0.5em}%
  \sffamily\begin{tabular}[t]{ll}}
  \postauthor{\end{tabular}\par\thispagestyle{plain}}

\predate{\noindent%
  \small\sffamily\itshape\begin{tabular}[t]{l}%
    EST-25134, Primavera 2021 \\ %
    Dr.\ Alfredo Garbuno Iñigo \\ %
  }
  \postdate{
  \end{tabular}\par}

%=============
% Section headings
%=============
\usepackage[sf,bf,compact]{titlesec}

%=============
% Tables and lists
%=============
\usepackage{booktabs,longtable,tabu} % Nice tables
\setlength{\tabulinesep}{1mm}
\usepackage[font=small,margin=10pt,labelfont={sf,bf},labelsep={space}]{caption}


\usepackage{enumitem}
\setitemize{itemsep=0pt}
\setenumerate{itemsep=0pt}
\setlist{labelindent=\parindent,%  % Recommended by enumitem package
  font=\sffamily}


%=============
% Hyperlink colors
%=============
\usepackage[usenames,dvipsnames]{xcolor}
\definecolor{dark-gray}{gray}{0.3}
\definecolor{dkgray}{rgb}{.4,.4,.4}
\definecolor{dkblue}{rgb}{0,0,.5}
\definecolor{medblue}{rgb}{0,0,.75}
\definecolor{rust}{rgb}{0.5,0.1,0.1}

\usepackage{url}
\usepackage[colorlinks=true]{hyperref}
\hypersetup{linkcolor=dkblue}
\hypersetup{citecolor=rust}
\hypersetup{urlcolor=rust}
\usepackage[spanish, capitalise]{cleveref}
\usepackage{autonum}
%=============
% Microtype
%=============
\usepackage[final]{microtype}

%=============
% Theorems, etc.
%=============
\newtheoremstyle{myThm} % name
    {\topsep}                    % Space above
    {\topsep}                    % Space below
    {\itshape}                   % Body font
    {}                           % Indent amount
    {\sffamily\bfseries}                   % Theorem head font
    {.}                          % Punctuation after theorem head
    {.5em}                       % Space after theorem head
    {}  % Theorem head spec (can be left empty, meaning ‘normal’)

\newtheoremstyle{myRem} % name
    {\topsep}                    % Space above
    {\topsep}                    % Space below
    {}                   % Body font
    {}                           % Indent amount
    {\sffamily}                   % Theorem head font
    {.}                          % Punctuation after theorem head
    {.5em}                       % Space after theorem head
    {}  % Theorem head spec (can be left empty, meaning ‘normal’)

\newtheoremstyle{myDef} % name
    {\topsep}                    % Space above
    {\topsep}                    % Space below
    {}                   % Body font
    {}                           % Indent amount
    {\sffamily\bfseries}                   % Theorem head font
    {.}                          % Punctuation after theorem head
    {.5em}                       % Space after theorem head
    {}  % Theorem head spec (can be left empty, meaning ‘normal’)

\theoremstyle{myThm}
\newtheorem{theorem}{Teorema}[section]
\newtheorem{lemma}[theorem]{Lema}
\newtheorem{proposition}[theorem]{Proposición}
\newtheorem{corollary}[theorem]{Corolario}
\newtheorem{fact}[theorem]{Dato}

\theoremstyle{myRem}
\newtheorem{remark}[theorem]{Observación}

\theoremstyle{myDef}
\newtheorem{definition}[theorem]{Definición}
\newtheorem{example}[theorem]{Ejemplo}

%=====================
% Header
%=====================
\usepackage{fancyhdr}
\usepackage{nopageno} % Gets rid of page number at the bottom
\fancyhf{} % Clear header style
\renewcommand{\headrulewidth}{0.5pt} % remove the header rule
\pagestyle{fancy}
\fancyhead[LE,RO]{\textsf{\small \thepage}}

\setlength{\headheight}{14pt}
%=====================
% Fix delimiters
%=====================

% Fixes \left and \right spacing issues. See discussion at
% http://tex.stackexchange.com/questions/2607/spacing-around-left-and-right
\let\originalleft\left
\let\originalright\right
\renewcommand{\left}{\mathopen{}\mathclose\bgroup\originalleft}
\renewcommand{\right}{\aftergroup\egroup\originalright}

%=================================================
% Math macros
%=================================================

%=============
% Generalities
%=============
\usepackage{mathtools}
\mathtoolsset{centercolon}  % Makes := typeset correctly for definitions

%%% Equation numbering
%\numberwithin{equation}{section}

%%% Annotations
\newcommand{\notate}[1]{\textcolor{red}{\textbf{[#1]}}}

%==============
% Symbols
%==============
\let\oldphi\phi
\let\oldeps\epsilon
\let\oldemptyset\emptyset
\let\emptyset\varnothing

\renewcommand{\phi}{\varphi}
\renewcommand{\epsilon}{\varepsilon}
\newcommand{\eps}{\varepsilon}
\newcommand{\cl}{\mathrm{cl}}
\newcommand{\wto}{\rightharpoonup}
\newcommand{\wsto}{\overset{\ast}{\rightharpoonup}}
\newcommand{\wwto}{\overset{w}{\to}}
\newcommand{\wwsto}{\overset{w*}{\to}}

%==============
% Constants
%==============

% Set constants upright
\newcommand{\cnst}[1]{\mathrm{#1}}
\newcommand{\econst}{\mathrm{e}}
\newcommand{\rd}{\mathrm{d}}
\newcommand{\dist}{\mathrm{dist}}

\newcommand{\zerovct}{\vct{0}} % Zero vector
\newcommand{\Id}{\mathbf{I}} % Identity matrix
\newcommand{\onemtx}{\bm{1}}
\newcommand{\zeromtx}{\bm{0}}

%==============
% Sets
%==============
\providecommand{\mathbbm}{\mathbb} % In case we don't load bbm

% Reals, complex, naturals, integers, field
\newcommand{\R}{\mathbbm{R}}
\newcommand{\C}{\mathbbm{C}}
\newcommand{\N}{\mathbbm{N}}
\newcommand{\Z}{\mathbbm{Z}}
\newcommand{\F}{\mathbbm{F}}

%==============
% Probability
%==============
\newcommand{\Prob}{\operatorname{\mathbbm{P}}}
\newcommand{\Expect}{\operatorname{\mathbb{E}}}
\newcommand{\D}{\operatorname{\mathcal{D}}}

%==============
% Vectors and matrices
%==============
\newcommand{\vct}[1]{\mathbold{#1}}
\newcommand{\mtx}[1]{\mathbold{#1}}

%=============
% Operators
%=============
\newcommand{\B}{\mathcal{B}}
\newcommand{\op}[1]{\mathbold{#1}}
\renewcommand{\H}{\mathcal{H}}
\newcommand{\X}{\mathcal{X}}
\newcommand{\Y}{\mathcal{Y}}
 % "macro.tex" must be in the same folder
\usepackage{bbm}
%>>>>>>> IF NEEDED, ADD A NEW FILE WITH YOUR OWN MACROS

% \input{lecture_01_macro.tex} % Name of supplemental macros should match lecture number

%>>>>>>> LECTURE NUMBER AND TITLE
\title{Clase 11/03:               % UPDATE LECTURE NUMBER
    Regularización y estabilidad}	% UPDATE TITLE
% TIP:  Use "\\" to break the title into more than one line.

%>>>>>>> DATE OF LECTURE
\date{Marzo 11, 2021} % Hard-code lecture date. Don't use "\today"

%>>>>>>> NAME OF SCRIBE(S)
\author{%
  Responsable:&
  Alejandro Chávez Mier  % >>>>> SCRIBE NAME(S)
}

\begin{document}
\maketitle %  LEAVE HERE
% The command above causes the title to be displayed.

%>>>>> DELETE ALL CONTENT UNTIL "\end{document}"
% This is the body of your document.


\begin{itemize}
    \item Hoy veremos que CLA y CSA son realmente logrables/aprendibles. Hay instancias de esos dos que son uniformes y por lo tanto aprendibles, con la regla más sencilla: ERM.
    \item Idea: tomamos una función de pérdida y sumamos algo que regularice.
    \item Nota: regularización nos sirve para medir la complejidad de la hipótesis y actúa como un estabilizador).
\end{itemize}

\section{Minimización de pérdida regularizada (RLM)}
RLM = pérdida empírica + función de regularización.
\\

El objetivo: $\underset{w \in H}{\mbox{argmin}} L_s(w)+R(w)$. Queremos, pues, encontrar un balance entre modelos simples y soluciones con error empírico pequeño.
$R()$ normalmente se escoje con conocimiento de dominio, y las opciones clásicas son $R(w) = \lambda*\|w\|^2_2$ ó $R(w) = \lambda*\|w\|^2_1$
\\

Regularización de Tikhonov (se usa para problemas inversos): supongamos que tenemos un problema de regresión de la forma $\frac{1}{2m}\|x_w-y\|^2 + \frac{\lambda}{2}\|w\|^2$, el cual buscamos minimizar.
Ya hemos visto que $\frac{1}{2m}\|x_w-y\|^2$ tiene un mínimo. La ventaja de $\frac{\lambda}{2}\|w\|^2$ es que no afecta mucho ese mínimo y, además, la solución de $\Delta_w(L_s(w)+R(w)=0$ es $(\lambda_m*\mathbbm{1}_{pxp})w = x^Ty*2w = (x^Tx +\lambda_m*\mathbbm{1}_{pxp})^{-1}x^Ty $.
Por ejemplo, si $P(w) = w_0+w_1x+\dots+w_px^p$ y $\lambda >> 0 $, esperaríamos que el modelo fuera parsimonioso y que los términos mayores se fueran eliminando.
\\

Verificaremos que $R(w)$ estabiliza y permite el sobreajuste. En particular, veremos un resultados en que 
\begin{center}
    $E_S(L_D(A(S))) \leq \underset{w \in H}{\min} L_D(w) + \varepsilon$
\end{center}

Otra forma de pensar el problema es: $\underset{w \in H}{\min} L_D(w)$ sujeto a $\|w\|^2_2 \leq \theta$

\section{Noción de estabilidad}
Sea $A$ un algoritmo de aprendizaje y sea $S (\thicksim D^m) = (Z_1,\dots, Z_m)$. Hablamos de sobreajuste cuando $|L_D(A(S))-L_S(A(S))|$ es grande.
Sea $S^{(i)} = (Z_1,\dots,Z_i,{Z'}, Z_{i+1},\dots,Z_m)$ con ${Z'}$ independiente de los anteriores y  ${Z'} \thicksim D^m$. Lo que esperaríamos de un buen argumento es:
\begin{center}
    $l(A(S^{(i)})),Z_i)*l(A(S)),Z_i) \geq 0$
\end{center}.

\begin{theorem}
Supongamos que $D$ es una distribución. Sea $S (\thicksim D^m) = (Z_1,\dots, Z_m)$, y sea ${Z'} \thicksim D^m$ una observación independiente. Denotamos como $U(m)$ a la distribución uniforme en el conjunto de índices $\{1,\dots,m\}$. Entonces

\begin{center}
$E_S(L_D(A(S)) - L_S(A(S))) = E_{\underset{i\thicksim U(m)}{S \thicksim D^m}}(l(A(S^{(i)})),Z_i) - l(A(S)),Z_i))$
\end{center}

\end{theorem}
\\

\begin{proof}
$E_S(L_D(A(S)) = E_{S,{Z'}}(l(A(S),{Z'})) = E_{\underset{i\thicksim U(m)}{S \thicksim D^m}}(l(A(S^{(i)})),Z_i)$
Por otro lado, $E_S(L_S(A(S)) = E_{S,i}(l(A(S),Z_i)) = E_S(\frac{1}{n} \sum_{i=1}^m l(A(S),Z_i)$
\end{proof}

\begin{definition}
Sea $\varepsilon :\mathbb{N} \rightarrow \mathbb{R}$ monótonamente decreciente. Decimos que el algoritmo $A$ es estable en promedio, bajo reemplazos individuales con tasa $\varepsilon(m)$, si $\forall D$ se tiene que $$E_{\underset{i\thicksim U(m)}{S \thicksim D^m}}(l(A(S^{(i)})),Z_i) - l(A(S)),Z_i)) \leq \varepsilon(m) $$
\end{definition}

\section{Regularización como estabilizador}

\begin{definition}
Una función $f$ es $\lambda$-fuertemente convexa si $\forall u,w$ y $\alpha \in (0,1)$ tenemos que 
$$f(\alpha w +(1-\alpha)u) \leq \alpha f(w) +(1-\alpha)f(u) - \frac{\lambda}{2} \alpha (1-\alpha)\|w-u\|^2$$
\end{definition}

\begin{lemma}
\begin{enumerate}
    \item $f(w) = \lambda\|w\|^2$ es $2\lambda$-fuertemente convexa.
    \item Si $f$ es $\lambda$-fuertemente convexa y $g$ es convexa $\Rightarrow f+g$ es fuertemente convexa.
    \item Si $f$ es $\lambda$-fuertemente convexa y $u$ es el minimizador de $f \Rightarrow \forall w, f(w)-f(u) \geq \frac{\lambda}{2}\|w-u\|^2$
\end{enumerate}
\end{lemma}

\begin{proof} Del inciso 3 del Lema 1.\\
$
f(u + \alpha(w-u)) - f(u) \leq \alpha f(w) -\alpha f(u) - \frac{\lambda}{2} \alpha (1-\alpha)\|w-u\|^2 \\
\Rightarrow
\frac{f(u + \alpha(w-u)) - f(u)}{\alpha} \leq f(w) - f(u) - \frac{\lambda}{2} (1-\alpha)\|w-u\|^2
$
Si $\alpha \rightarrow 0$, el término de la derecha equivale a la derivada evaluada en el minimizados
\end{proof}
\\

Falta ver que RLM es estable. Consideremos $S, {Z'}, S^{(i)}$ como arriba, y $A$ RLM. Entonces
\begin{center}
$
A(S) = \underset{w \in H}{\mbox{argmin}} L_s(w) + \lambda\|w\|^2$ y 
$f_S(w) = L_S(w) +\lambda\|w\|^2$ ($2\lambda$-fuertemente convexa).
\end{center}
Además, $f_S(w) - f_S(A(s)) \geq \lambda \|v-A(S)\|^2$

\begin{equation*}
\Rightarrow f_S(w) -f_S(u) = L_S(v) + \lambda \|v\|^2 - L_S(u) - \lambda\|u\|^2 \\
L_S(u) = L_{S^{(i)}}(v) + \frac{l(v,Z_i) - l(v, {Z'})}{m}

\therefore \lambda \|A(S^{(i)}) - A(S)\|^2 \leq f_S(A(S^{(i)})) - f_S(A(S)) \leq \frac{l(A(S^{(i)}),Z_i) - l(A(S), Z_i)}{m} + \frac{l(A(S),{Z'}) - l(A(S^{(i)}), {Z'})}{m}
\end{equation*}

\subsection{Bajo Lipschitz}
\begin{equation*}
l(A(S^{(i)}),Z_i) - l(A(S), Z_i) \leq \rho \|A(S^{(i)}) - A(S) \|
\Rightarrow \|A(S^{(i)}) - A(S) \| \leq \frac{2\rho}{\lambda m}
\end{equation*}

\begin{equation*}
\Rightarrow E_S(L_D(A(S)) - L_S(A(S))) \leq \frac{2\rho^2}{\lambda m}
\end{equation*}

\section{Control de sobreajuste y estabilidad}
$E_S(L_D(A(S))) = E_S(L_S(A(S))) + E_S(L_D(A(S))-L_S(A(S))$
Si $\lambda$ crece, el error empírico también.
\\

Dado que $A(S) = \mbox{argmin} L_S(w) + \lambda\|w\|^2$,
\begin{equation*}
    L_S(A(S)) \leq L_S(A(S)) + \lambda \|A(S)\|^2 \leq L_S(w) + \lambda \|w\|^2
\end{equation*}
\begin{equation*}
    \Rightarrow E_S(L_S(A(S))) \leq L_D(w) + \lambda \|w\|^2, E_S(L_D(A(S))) \leq L_D(w) + \lambda \|w\|^2 + estabilidad
\end{equation*}

\begin{equation*}
    \therefore E_S(L_D(A(S))) \leq L_D(w) + \lambda \|w\|^2 + \frac{2 \rho}{\lambda m}
\end{equation*}








   
%>>>>>> END OF YOUR CONTENT

%>>>>>>>>>> take acknowledgements out
\section*{Agradecimientos} Este {\emph template} se ha adaptado y traducido del
provisto en la clase ACM 204 (Otoño 2017) por el profesor Joel Tropp.


\bibliographystyle{siam} % <<< USE "alpha" BIBLIOGRAPHY STYLE
\bibliography{template} % <<< RENAME TO "lecture_XX"


\end{document}

