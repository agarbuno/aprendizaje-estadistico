\documentclass[11pt,reqno,twoside]{article}
%>>>>>>> RENAME CURRENT FILE TO MATCH LECTURE NUMBER
% E.g., "lecture_01.tex"

%>>>>>>> DO NOT EDIT MACRO FILE

% >>>> DO NOT EDIT THIS FILE
% If you *must* add or change a macro, please email fkoh@caltech.edu

%=================================================
% Basics
%=================================================

\usepackage{fixltx2e} % Makes \( \) equation style robust, among other
                      % things. Must be the first package.


% Makes ligatured fonts searchable and copyable in pdf readers
\usepackage{cmap} % Load before fontenc
\usepackage[spanish]{babel}

% Always include these font encodings in your document
% unless you have a very good reason.
\usepackage[T1]{fontenc}
\usepackage[utf8]{inputenc}

\usepackage{verbatim}

%=====================================
% Look & feel
%=====================================

% Allows for manual space setting
\usepackage{setspace}

%=============
% Fonts
%=============

\usepackage{lmodern} % Improved version of computer modern
\usepackage[scale=0.88]{tgheros} % Helvetica clone for sans serif font


\newcommand\hmmax{2} % Default is 3.
\newcommand\bmmax{2} % Default is 4.

\usepackage{bm} % boldmath must be called after the package
\providecommand{\mathbold}[1]{\bm{#1}}
\usepackage{dsfont}
%=============
% AMS Packages and fonts
%=============
\usepackage{amsmath,amsbsy,amsgen,amscd,amsthm,amsfonts,amssymb}

%=============
% Margins and paper size
%=============
\usepackage[centering,top=1.5in,bottom=1.2in,left=1.4in,right=1.4in]{geometry}

%=============
% Title setup
%=============
\usepackage{titling}
%\usepackage{nopageno}
\setlength{\droptitle}{-7.5em}

\pretitle{\noindent\rule{0.85\linewidth}{0.2mm}\par%
  \begin{raggedright}\LARGE\sffamily}
\posttitle{\par\end{raggedright}%
\noindent%
\rule{0.85\linewidth}{0.5mm}\par}

\preauthor{\noindent\vspace{0.5em}%
  \sffamily\begin{tabular}[t]{ll}}
  \postauthor{\end{tabular}\par\thispagestyle{plain}}

\predate{\noindent%
  \small\sffamily\itshape\begin{tabular}[t]{l}%
    EST-25134, Primavera 2021 \\ %
    Dr.\ Alfredo Garbuno Iñigo \\ %
  }
  \postdate{
  \end{tabular}\par}

%=============
% Section headings
%=============
\usepackage[sf,bf,compact]{titlesec}

%=============
% Tables and lists
%=============
\usepackage{booktabs,longtable,tabu} % Nice tables
\setlength{\tabulinesep}{1mm}
\usepackage[font=small,margin=10pt,labelfont={sf,bf},labelsep={space}]{caption}


\usepackage{enumitem}
\setitemize{itemsep=0pt}
\setenumerate{itemsep=0pt}
\setlist{labelindent=\parindent,%  % Recommended by enumitem package
  font=\sffamily}


%=============
% Hyperlink colors
%=============
\usepackage[usenames,dvipsnames]{xcolor}
\definecolor{dark-gray}{gray}{0.3}
\definecolor{dkgray}{rgb}{.4,.4,.4}
\definecolor{dkblue}{rgb}{0,0,.5}
\definecolor{medblue}{rgb}{0,0,.75}
\definecolor{rust}{rgb}{0.5,0.1,0.1}

\usepackage{url}
\usepackage[colorlinks=true]{hyperref}
\hypersetup{linkcolor=dkblue}
\hypersetup{citecolor=rust}
\hypersetup{urlcolor=rust}
\usepackage[spanish, capitalise]{cleveref}
\usepackage{autonum}
%=============
% Microtype
%=============
\usepackage[final]{microtype}

%=============
% Theorems, etc.
%=============
\newtheoremstyle{myThm} % name
    {\topsep}                    % Space above
    {\topsep}                    % Space below
    {\itshape}                   % Body font
    {}                           % Indent amount
    {\sffamily\bfseries}                   % Theorem head font
    {.}                          % Punctuation after theorem head
    {.5em}                       % Space after theorem head
    {}  % Theorem head spec (can be left empty, meaning ‘normal’)

\newtheoremstyle{myRem} % name
    {\topsep}                    % Space above
    {\topsep}                    % Space below
    {}                   % Body font
    {}                           % Indent amount
    {\sffamily}                   % Theorem head font
    {.}                          % Punctuation after theorem head
    {.5em}                       % Space after theorem head
    {}  % Theorem head spec (can be left empty, meaning ‘normal’)

\newtheoremstyle{myDef} % name
    {\topsep}                    % Space above
    {\topsep}                    % Space below
    {}                   % Body font
    {}                           % Indent amount
    {\sffamily\bfseries}                   % Theorem head font
    {.}                          % Punctuation after theorem head
    {.5em}                       % Space after theorem head
    {}  % Theorem head spec (can be left empty, meaning ‘normal’)

\theoremstyle{myThm}
\newtheorem{theorem}{Teorema}[section]
\newtheorem{lemma}[theorem]{Lema}
\newtheorem{proposition}[theorem]{Proposición}
\newtheorem{corollary}[theorem]{Corolario}
\newtheorem{fact}[theorem]{Dato}

\theoremstyle{myRem}
\newtheorem{remark}[theorem]{Observación}

\theoremstyle{myDef}
\newtheorem{definition}[theorem]{Definición}
\newtheorem{example}[theorem]{Ejemplo}

%=====================
% Header
%=====================
\usepackage{fancyhdr}
\usepackage{nopageno} % Gets rid of page number at the bottom
\fancyhf{} % Clear header style
\renewcommand{\headrulewidth}{0.5pt} % remove the header rule
\pagestyle{fancy}
\fancyhead[LE,RO]{\textsf{\small \thepage}}

\setlength{\headheight}{14pt}
%=====================
% Fix delimiters
%=====================

% Fixes \left and \right spacing issues. See discussion at
% http://tex.stackexchange.com/questions/2607/spacing-around-left-and-right
\let\originalleft\left
\let\originalright\right
\renewcommand{\left}{\mathopen{}\mathclose\bgroup\originalleft}
\renewcommand{\right}{\aftergroup\egroup\originalright}

%=================================================
% Math macros
%=================================================

%=============
% Generalities
%=============
\usepackage{mathtools}
\mathtoolsset{centercolon}  % Makes := typeset correctly for definitions

%%% Equation numbering
%\numberwithin{equation}{section}

%%% Annotations
\newcommand{\notate}[1]{\textcolor{red}{\textbf{[#1]}}}

%==============
% Symbols
%==============
\let\oldphi\phi
\let\oldeps\epsilon
\let\oldemptyset\emptyset
\let\emptyset\varnothing

\renewcommand{\phi}{\varphi}
\renewcommand{\epsilon}{\varepsilon}
\newcommand{\eps}{\varepsilon}
\newcommand{\cl}{\mathrm{cl}}
\newcommand{\wto}{\rightharpoonup}
\newcommand{\wsto}{\overset{\ast}{\rightharpoonup}}
\newcommand{\wwto}{\overset{w}{\to}}
\newcommand{\wwsto}{\overset{w*}{\to}}

%==============
% Constants
%==============

% Set constants upright
\newcommand{\cnst}[1]{\mathrm{#1}}
\newcommand{\econst}{\mathrm{e}}
\newcommand{\rd}{\mathrm{d}}
\newcommand{\dist}{\mathrm{dist}}

\newcommand{\zerovct}{\vct{0}} % Zero vector
\newcommand{\Id}{\mathbf{I}} % Identity matrix
\newcommand{\onemtx}{\bm{1}}
\newcommand{\zeromtx}{\bm{0}}

%==============
% Sets
%==============
\providecommand{\mathbbm}{\mathbb} % In case we don't load bbm

% Reals, complex, naturals, integers, field
\newcommand{\R}{\mathbbm{R}}
\newcommand{\C}{\mathbbm{C}}
\newcommand{\N}{\mathbbm{N}}
\newcommand{\Z}{\mathbbm{Z}}
\newcommand{\F}{\mathbbm{F}}

%==============
% Probability
%==============
\newcommand{\Prob}{\operatorname{\mathbbm{P}}}
\newcommand{\Expect}{\operatorname{\mathbb{E}}}
\newcommand{\D}{\operatorname{\mathcal{D}}}

%==============
% Vectors and matrices
%==============
\newcommand{\vct}[1]{\mathbold{#1}}
\newcommand{\mtx}[1]{\mathbold{#1}}

%=============
% Operators
%=============
\newcommand{\B}{\mathcal{B}}
\newcommand{\op}[1]{\mathbold{#1}}
\renewcommand{\H}{\mathcal{H}}
\newcommand{\X}{\mathcal{X}}
\newcommand{\Y}{\mathcal{Y}}
 % "macro.tex" must be in the same folder

%>>>>>>> IF NEEDED, ADD A NEW FILE WITH YOUR OWN MACROS

% \input{lecture_01_macro.tex} % Name of supplemental macros should match lecture number

%>>>>>>> LECTURE NUMBER AND TITLE
\title{Clase 07:               % UPDATE LECTURE NUMBER
    Modelos Lineales}	% UPDATE TITLE
% TIP:  Use "\\" to break the title into more than one line.

%>>>>>>> DATE OF LECTURE
\date{Febrero 04, 2021} % Hard-code lecture date. Don't use "\today"

%>>>>>>> NAME OF SCRIBE(S)
\author{%
  Responsable:&
  Jordi Legorreta Montaño  % >>>>> SCRIBE NAME(S)
}

\begin{document}
\maketitle %  LEAVE HERE
% The command above causes the title to be displayed.

%>>>>> DELETE ALL CONTENT UNTIL "\end{document}"
% This is the body of your document.

\section{Modelos Lineales}
\label{sec:ML}

\begin{description}\setlength{\itemsep}{0pt} % Compact description list
    \item [Antes]: modelo de clasificación por semiespacios
    \begin{center}
        \begin{minipage}[c]{0.7\linewidth}
            (También conocido como discriminación lineal).\\
            Algoritmo para entrenarlo: perceptrones.\\
            \textbf{NOTA}: No confundir modelo con algoritmo.\\
            $\hat{y}\in\{0,1\}$\\
            $y\langle w,x\rangle \geq1\hspace{2mm}\leftarrow$\hspace{2mm} margen de clasificación\\
            $y\langle w,x\rangle>0$\\
            $w^{(t+1)}=w^{(t)}+y^{(i)}x^{(i)}$
        \end{minipage}
    \end{center}
\end{description}

El algoritmo, lo que busca, es que después de cada iteración, los casos mal clasificados sean bien clasificados.

$$y^{(i)}\langle w^{(t+1)},x^{(i)}\rangle\geq0

Dado el criterio anterior, ¿cuál será la función de pérdida?\\

\subsection{Modelo de Regresión}

Sirven para describir una relación entre un conjunto de características $x\in\mathcal{X}\subseteq\mathbb{R}^{p}$, y la respuesta (en este caso) será $y\in\mathbb{R}$.

\begin{description}\setlength{\itemsep}{0pt} % Compact description list
    \item [x]: variables independientes, explicativas o exploratorias.
    \item[y]: variable dependiente o variable objeto.
    \begin{center}
        \begin{minipage}[c]{0.7\linewidth}
            $y=f(x)\hspace{2mm}\leftarrow\hspace{2mm}$ relación entre $\mathcal{X}$ e $\mathcal{Y}$
        \end{minipage}
    \end{center}
\end{description}

Queremos encontrar $h:\mathbb{R}^{d}\to\mathbb{R}$, la mejor aproximación a dicha relación.

\begin{center}
    \begin{minipage}[c]{0.7\linewidth}
        $\mathcal{H}_{reg}=L_{p}=\{h_{w,b}:w\in\mathbb{R}^{p},\hspace{2mm}b\in\mathbb{R},\hspace{2mm}h_{w,b}(x)=\langle x,w\rangle+b\}$\\
        
        La ``mejor'' $h\in\mathcal{H}_{reg}$ es la que minimice el error:\\
        
        $l(\hat{y},y)=\frac{1}{2}(\hat{y}-y)^2$\\
        $l(w,x,y)=\frac{1}{2}(\langle w,x\rangle-y)^2$,\hspace{2mm} $w\in\mathbb{R}^{p+1}$\\
        
        $L_{s}(h)=L_{s}(w)=\frac{1}{2m}\sum_{i=1}^{m}(\langle w,x^{(i)}\rangle-y^{(i)})^2$
    \end{minipage}
\end{center}

Con la función de perdida anterior, el principio de minimización de riesgo empírico nos dice que:

\begin{center}
    \begin{minipage}[c]{0.7\linewidth}
        $ERM\to h^{*}=h_{w}^{*}\mid w^{*}=$ argmin $L_{s}(w)$\\
        
        $L_{s}(w)=\frac{1}{2m}\|\mathbb{X}w-y\|^2$\\
        
        $\mathbb{X}$ &= \begin{bmatrix}
            -$X^{(1)T}-$ \\
            -$X^{(2)T}-$ \\
            \vdots \\
            -$X^{(m)T}-$
            \end{bmatrix}
            \in\mathbb{R}^{mxp}, \hspace{4mm}
        y &= \begin{bmatrix}
            y^{1} \\
            y^{2} \\
            \vdots \\
            y^{m}
            \end{bmatrix}
            \in\mathbb{R}^{m}
    \end{minipage}
\end{center}

Ventajas de la función $L_{s}(w)$:
\begin{itemize}
    \item Función diferenciable, por lo que podemos calcular la función gradiente parametrizada por $w$.
\end{itemize}

$$\nabla_{w}L_{s}(w)=\nabla_{w}(\frac{1}{2m}\|\mathbb{X}w-y\|^2)=\frac{2}{2m}(\mathbb{X}w-y)^{T}\mathbb{X}$$

$$\nabla_{w}L_{s}(w)=0\iff\frac{1}{m}(\mathbb{X}w-y)^{T}\mathbb{X}=0\iff w^{*}=(\mathbb{X}^{T}\mathbb{X})^{-1}\mathbb{X}^{T}y$$

$$w^{*}=A^{-1}b\iff Aw^{*}=b$$

¿Qué condiciones debe satisfacer la matriz A?

\begin{itemize}
    \item Que sea invertible
    \item Que sea positiva definida
    \begin{enumerate}[label=(\roman*)]
        \item $m>p$
        \item No haya atributos linealmente dependientes
    \end{enumerate}
    \item $A^{\dagger}$ es la pseudoinversa
    \begin{center}
    \begin{minipage}[c]{0.7\linewidth}
        $A=vDv^{T}$,\hspace{2mm} donde $vv^{T}=v_{T}v=\mathbb{I}_{p}$\\
        $A^{\dagger}=vD^{\dagger}v^{T}$\\
        $D=diag(d_{1},d_{2},...,d_{p})$\\
        $D^{\dagger}=diag(1/d_{1},...,1/d_{k},0,...,0)$
    \end{minipage}
\end{center}
\end{itemize}

$ERM\hspace{2mm}\rightarrow\hspace{2mm}$ solución: $w^{*}=A^{\dagger}x^{T}y$

Concluyendo: este problema de regresión tiene una solución y se puede encontrar de manera analítica.

\subsection{Conectar ERM con estimadores de máxima verosimilitud}

\begin{description}\setlength{\itemsep}{0pt} % Compact description list
    \item [$y=h_{w}(x)+\epsilon$]: consideramos cierta desviación
    \begin{center}
        \begin{minipage}[c]{0.7\linewidth}
            $\epsilon\sim N(0,\sigma^{2})$\\
            $\epsilon$: error de estimación, discrepancia que existe entre el modelo lineal y los datos.\\
            $y=\hat{y}+\epsilon$
        \end{minipage}
    \end{center}
    \item [Definición]: $\pi(y\mid x,w)=\frac{1}{\sqrt{2\sigma^{2}\pi}}e^{-\frac{1}{2\sigma^{2}}(y-\langle w,x\rangle)^2}$,\hspace{2mm} donde\hspace{2mm} $y\sim N(\langle w,x\rangle,\sigma^{2})$
    \begin{center}
        \begin{minipage}[c]{0.7\linewidth}
            $log\pi(y\mid x,w)=-\frac{1}{2\sigma^{2}}(y-\langle w,x\rangle)^2+c$\\
            $l(\hat{y},y)=-log\pi(y\mid x,w)$\\
            $L_{s}(w)=\frac{1}{m}\sum_{i=1}^{m}l(\hat{y}^{(i)},y^{(i)})=-\frac{1}{m}\sum_{i=1}^{m}log\pi(y\mid x,w)$\\
            $-\frac{1}{m}\sum_{i=1}^{m}log\pi(y\mid x,w)=-l_{m}(w)$\\
            $\mathcal{L}_{m}(w)=\Pi_{i=1}^{m}\pi(y^{(i)}\mid x^{(i)},w)\hspace{2mm}\leftarrow\hspace{2mm} verosimilitud$\\
            $l_{m}(w)=log(\mathcal{L}_{m}(w))=\frac{1}{m}\sum_{i=1}^{m}log\pi(y^{(i)}\mid x^{(i)},w)$\\
            $\frac{1}{m}\sum_{i=1}^{m}log\pi(y^{(i)}\mid x^{(i)},w)\hspace{2mm}\leftarrow\hspace{2mm} log-verosimilitud$\\
            $w_{ERM}^{*}=$ argmin $L_{s}(w)=$ argmax $l_{m}(w)=\hat{w}^{MLE}$
        \end{minipage}
    \end{center}
\end{description}

En el contexto de regresión lineal con errores gaussianos, el principio de minimización de riesgo empírico es equivalente a estimar un parámetro como si fuera un problema estadístico de máxima verosimilitud.

\subsection{Conexión de ERM con Teoría de la Información}

\begin{description}\setlength{\itemsep}{0pt} % Compact description list
    \item [Teoría de la información]: Codificar, decodificar, transmitir y manipular paquetes de información en canales de comunicación\cite{mackay}.
\end{description}

Lo que más nos interesa en Teoría de la Información es:

\begin{enumerate}[label=(\roman*)]
    \item Entropía: $H(P)=\sum_{j}-p_{j}log(p_{j})$, la información con la que podemos codificar la realización de un evento aleatorio a través de una distribución $P$.\\
    "Sorpresas": los eventos de probabilidad muy bajos, cuando ocurren, nos sorprenden, porque no los esperamos.
    $$log(\frac{1}{p_{j}})=-log(p_{j})$$\\
    De alguna manera la entropía es el valor esperado de la sorpresa.
    \item Entropía cruzada: $H(P,Q)=\sum_{j}-p_{j}log(q_{j})$, los eventos aleatorios se están generando con respecto a una distribución $P$ y los estamos codificando utilizando una distribución $Q$.\\
    Por la desigualdad de Jensen, tenemos que: $H(P)=H(P,P)\leq H(P,Q)$
    \item Entropía cruzada relativa: $D_{kl}(P\|Q)=H(P,Q)-H(P,P)\geq0$, donde $D_{kl}$ se conoce como "Divergencia" de Kullback-Leibler
\end{enumerate}

Entonces: $L_{s}(h)=\sum_{i=1}^{m}-log\pi(y^{(i)}\mid x^{(i)},w)\frac{1}{m}=H(D^{m},D_{w}^{m})$, entropía cruzada entre la distribución empírica que generan los datos $D^{m}$ y una distribución de $m$ datos utilizando una codificación basada en $w$.

Por lo tanto, $w^{*}=$ argmin $L_{s}(h)=$ argmin $H(D^{m},D_{w}^{m})$

\subsection{Familias Polinomiales}

$P(x)=a_{0}+a_{1}x+...+a_{p}x^{p}=\langle a,\Phi(x)\rangle$, donde $a=(a_{0},a_{1},...,a_{p})^{T}\in\mathbb{R}^{p+1}$ y $\Phi(x)=(1,x,...,x^{p})\in\mathbb{R}^{p+1}$.

$a^{*}\hspace{2mm}\leftarrow\hspace{2mm}$ utilizamos $ERM$ con pérdida cuadrática.

Ya no son sólo rectas, sino que ahora trabajamos con polinomios.

\section{Modelo de Regresión en clasificación (binaria)}

\begin{description}\setlength{\itemsep}{0pt} % Compact description list
    $h:\mathbb{R}^{p}\to[0,1]$, donde $h(x)$ es la probabilidad de obtener $y=1$.\\
    $\sigma(x)=\frac{e^{x}}{1+e^{x}}=\frac{1}{1+e^{-x}}$, donde $\sigma(x)$ es la función sigmoide, la cual nos ayuda a comprimir la recta real, de tal forma que este problema de usar modelos de regresión para clasificación le llamamos:
\end{description}

\textbf{Regresión logística}: $\mathcal{H}_{log}=\sigma_{w}L_{p}\{\sigma_{w}:w\in\mathbb{R}^{p},\hspace{2mm}\sigma_{w}(x)=\sigma(\langle x,w\rangle)\}$

\begin{itemize}
    \item Si $\langle w,x\rangle$ es muy grande, entonces $\sigma_{w}(x)\approx1$
    \item Si $\langle w,x\rangle$ es muy pequeño (en términos negativos), entonces $\sigma_{w}(x)\approx0$
    \item Si $\langle w,x\rangle=0$, entonces $\sigma_{w}(x)=\frac{1}{2}$
\end{itemize}

¿Qué función de pérdida utilizar?

\begin{description}\setlength{\itemsep}{0pt} % Compact description list
    $\sigma_{w}(x)=\mathbb{P}(y=1\mid x,w)\implies y\sim Bernoulli(\sigma_{w}(x))$\\
    Verosimilitud de una Bernoulli: $(\sigma_{w}(x))^{y}(1-\sigma_{w}(x))^{1-y}$\\
    log-verosimilitud: $ylog(\sigma_{w})+(1-y)log(1-\sigma_{w})$
\end{description}

Si $\sigma_{w}(x)=\frac{e^{w^{T}x}}{1+e^{w^{T}}x}$, entonces:

\begin{description}\setlength{\itemsep}{0pt} % Compact description list
    log-verosimilitud: $-log(1+e^{-y\langle w,x\rangle})$,\hspace{2mm} donde $y\in\{1,-1\}$\\
    $l(\hat{y},y)=log(1+e^{-\hat{y}\langle w,x\rangle})$,\hspace{2mm} donde $\hat{y}=\sigma_{w}(x)$\\
    $L_{s}(w)=\frac{1}{m}\sum_{i=1}^{m}log(1+e^{-y^{(i)}\langle w,x^{(i)}\rangle})$
\end{description}


%>>>>>> END OF YOUR CONTENT

%>>>>>>>>>> take acknowledgements out
\section*{Agradecimientos} Este {\emph template} se ha adaptado y traducido del
provisto en la clase ACM 204 (Otoño 2017) por el profesor Joel Tropp.


%\bibliographystyle{siam} % <<< USE "alpha" BIBLIOGRAPHY STYLE
%\bibliography{lecture_07} % <<< RENAME TO "lecture_XX"

\begin{thebibliography}{}

\bibitem{mackay}
Mackay 2003: Information Theory, Inference and Learning.

\end{thebibliography}


\end{document}