\documentclass[11pt,reqno,twoside]{article}
%>>>>>>> RENAME CURRENT FILE TO MATCH LECTURE NUMBER
% E.g., "lecture_01.tex"

%>>>>>>> DO NOT EDIT MACRO FILE

% >>>> DO NOT EDIT THIS FILE
% If you *must* add or change a macro, please email fkoh@caltech.edu

%=================================================
% Basics
%=================================================

\usepackage{fixltx2e} % Makes \( \) equation style robust, among other
                      % things. Must be the first package.


% Makes ligatured fonts searchable and copyable in pdf readers
\usepackage{cmap} % Load before fontenc
\usepackage[spanish]{babel}

% Always include these font encodings in your document
% unless you have a very good reason.
\usepackage[T1]{fontenc}
\usepackage[utf8]{inputenc}

\usepackage{verbatim}

%=====================================
% Look & feel
%=====================================

% Allows for manual space setting
\usepackage{setspace}

%=============
% Fonts
%=============

\usepackage{lmodern} % Improved version of computer modern
\usepackage[scale=0.88]{tgheros} % Helvetica clone for sans serif font


\newcommand\hmmax{2} % Default is 3.
\newcommand\bmmax{2} % Default is 4.

\usepackage{bm} % boldmath must be called after the package
\providecommand{\mathbold}[1]{\bm{#1}}
\usepackage{dsfont}
%=============
% AMS Packages and fonts
%=============
\usepackage{amsmath,amsbsy,amsgen,amscd,amsthm,amsfonts,amssymb}

%=============
% Margins and paper size
%=============
\usepackage[centering,top=1.5in,bottom=1.2in,left=1.4in,right=1.4in]{geometry}

%=============
% Title setup
%=============
\usepackage{titling}
%\usepackage{nopageno}
\setlength{\droptitle}{-7.5em}

\pretitle{\noindent\rule{0.85\linewidth}{0.2mm}\par%
  \begin{raggedright}\LARGE\sffamily}
\posttitle{\par\end{raggedright}%
\noindent%
\rule{0.85\linewidth}{0.5mm}\par}

\preauthor{\noindent\vspace{0.5em}%
  \sffamily\begin{tabular}[t]{ll}}
  \postauthor{\end{tabular}\par\thispagestyle{plain}}

\predate{\noindent%
  \small\sffamily\itshape\begin{tabular}[t]{l}%
    EST-25134, Primavera 2021 \\ %
    Dr.\ Alfredo Garbuno Iñigo \\ %
  }
  \postdate{
  \end{tabular}\par}

%=============
% Section headings
%=============
\usepackage[sf,bf,compact]{titlesec}

%=============
% Tables and lists
%=============
\usepackage{booktabs,longtable,tabu} % Nice tables
\setlength{\tabulinesep}{1mm}
\usepackage[font=small,margin=10pt,labelfont={sf,bf},labelsep={space}]{caption}


\usepackage{enumitem}
\setitemize{itemsep=0pt}
\setenumerate{itemsep=0pt}
\setlist{labelindent=\parindent,%  % Recommended by enumitem package
  font=\sffamily}


%=============
% Hyperlink colors
%=============
\usepackage[usenames,dvipsnames]{xcolor}
\definecolor{dark-gray}{gray}{0.3}
\definecolor{dkgray}{rgb}{.4,.4,.4}
\definecolor{dkblue}{rgb}{0,0,.5}
\definecolor{medblue}{rgb}{0,0,.75}
\definecolor{rust}{rgb}{0.5,0.1,0.1}

\usepackage{url}
\usepackage[colorlinks=true]{hyperref}
\hypersetup{linkcolor=dkblue}
\hypersetup{citecolor=rust}
\hypersetup{urlcolor=rust}
\usepackage[spanish, capitalise]{cleveref}
\usepackage{autonum}
%=============
% Microtype
%=============
\usepackage[final]{microtype}

%=============
% Theorems, etc.
%=============
\newtheoremstyle{myThm} % name
    {\topsep}                    % Space above
    {\topsep}                    % Space below
    {\itshape}                   % Body font
    {}                           % Indent amount
    {\sffamily\bfseries}                   % Theorem head font
    {.}                          % Punctuation after theorem head
    {.5em}                       % Space after theorem head
    {}  % Theorem head spec (can be left empty, meaning ‘normal’)

\newtheoremstyle{myRem} % name
    {\topsep}                    % Space above
    {\topsep}                    % Space below
    {}                   % Body font
    {}                           % Indent amount
    {\sffamily}                   % Theorem head font
    {.}                          % Punctuation after theorem head
    {.5em}                       % Space after theorem head
    {}  % Theorem head spec (can be left empty, meaning ‘normal’)

\newtheoremstyle{myDef} % name
    {\topsep}                    % Space above
    {\topsep}                    % Space below
    {}                   % Body font
    {}                           % Indent amount
    {\sffamily\bfseries}                   % Theorem head font
    {.}                          % Punctuation after theorem head
    {.5em}                       % Space after theorem head
    {}  % Theorem head spec (can be left empty, meaning ‘normal’)

\theoremstyle{myThm}
\newtheorem{theorem}{Teorema}[section]
\newtheorem{lemma}[theorem]{Lemma}
\newtheorem{proposition}[theorem]{Proposition}
\newtheorem{corollary}[theorem]{Corollary}
\newtheorem{fact}[theorem]{Fact}

\theoremstyle{myRem}
\newtheorem{remark}[theorem]{Remark}

\theoremstyle{myDef}
\newtheorem{definition}[theorem]{Definition}
\newtheorem{example}[theorem]{Example}

%=====================
% Header
%=====================
\usepackage{fancyhdr}
\usepackage{nopageno} % Gets rid of page number at the bottom
\fancyhf{} % Clear header style
\renewcommand{\headrulewidth}{0.5pt} % remove the header rule
\pagestyle{fancy}
\fancyhead[LE,RO]{\textsf{\small \thepage}}

\setlength{\headheight}{14pt}
%=====================
% Fix delimiters
%=====================

% Fixes \left and \right spacing issues. See discussion at
% http://tex.stackexchange.com/questions/2607/spacing-around-left-and-right
\let\originalleft\left
\let\originalright\right
\renewcommand{\left}{\mathopen{}\mathclose\bgroup\originalleft}
\renewcommand{\right}{\aftergroup\egroup\originalright}

%=================================================
% Math macros
%=================================================

%=============
% Generalities
%=============
\usepackage{mathtools}
\mathtoolsset{centercolon}  % Makes := typeset correctly for definitions

%%% Equation numbering
%\numberwithin{equation}{section}

%%% Annotations
\newcommand{\notate}[1]{\textcolor{red}{\textbf{[#1]}}}

%==============
% Symbols
%==============
\let\oldphi\phi
\let\oldeps\epsilon
\let\oldemptyset\emptyset
\let\emptyset\varnothing

\renewcommand{\phi}{\varphi}
\renewcommand{\epsilon}{\varepsilon}
\newcommand{\eps}{\varepsilon}
\newcommand{\cl}{\mathrm{cl}}
\newcommand{\wto}{\rightharpoonup}
\newcommand{\wsto}{\overset{\ast}{\rightharpoonup}}
\newcommand{\wwto}{\overset{w}{\to}}
\newcommand{\wwsto}{\overset{w*}{\to}}

%==============
% Constants
%==============

% Set constants upright
\newcommand{\cnst}[1]{\mathrm{#1}}
\newcommand{\econst}{\mathrm{e}}
\newcommand{\rd}{\mathrm{d}}
\newcommand{\dist}{\mathrm{dist}}

\newcommand{\zerovct}{\vct{0}} % Zero vector
\newcommand{\Id}{\mathbf{I}} % Identity matrix
\newcommand{\onemtx}{\bm{1}}
\newcommand{\zeromtx}{\bm{0}}

%==============
% Sets
%==============
\providecommand{\mathbbm}{\mathbb} % In case we don't load bbm

% Reals, complex, naturals, integers, field
\newcommand{\R}{\mathbbm{R}}
\newcommand{\C}{\mathbbm{C}}
\newcommand{\N}{\mathbbm{N}}
\newcommand{\Z}{\mathbbm{Z}}
\newcommand{\F}{\mathbbm{F}}

%==============
% Probability
%==============
\newcommand{\Prob}{\operatorname{\mathbbm{P}}}
\newcommand{\Expect}{\operatorname{\mathbb{E}}}
\newcommand{\D}{\operatorname{\mathcal{D}}}

%==============
% Vectors and matrices
%==============
\newcommand{\vct}[1]{\mathbold{#1}}
\newcommand{\mtx}[1]{\mathbold{#1}}

%=============
% Operators
%=============
\newcommand{\B}{\mathcal{B}}
\newcommand{\op}[1]{\mathbold{#1}}
\renewcommand{\H}{\mathcal{H}}
\newcommand{\X}{\mathcal{X}}
\newcommand{\Y}{\mathcal{Y}}
 % "macro.tex" must be in the same folder

%>>>>>>> IF NEEDED, ADD A NEW FILE WITH YOUR OWN MACROS

% \input{lecture_01_macro.tex} % Name of supplemental macros should match lecture number

%>>>>>>> LECTURE NUMBER AND TITLE
\title{Clase 18:               % UPDATE LECTURE NUMBER
    Aprendizaje Estadístico}	% UPDATE TITLE
% TIP:  Use "\\" to break the title into more than one line.

%>>>>>>> DATE OF LECTURE
\date{23 marzo 2021} % Hard-code lecture date. Don't use "\today"

%>>>>>>> NAME OF SCRIBE(S)
\author{%
  Responsable:&
  Francisco Velasco Medina  % >>>>> SCRIBE NAME(S)
}

\begin{document}
\maketitle %  LEAVE HERE
% The command above causes the title to be displayed.

%>>>>> DELETE ALL CONTENT UNTIL "\end{document}"
% This is the body of your document.

\section{Generalización de Máquinas de Soporte Vectorial}
\label{sec:introduction}

%imagen

\begin{definition}{Definición}

Un \textbf{espacio de Hilbert} es un espacio vectorial
con producto interior y completo.

\end{definition}

%imagen

Llevamos los datos de $\psi(x) = (x, x^{2})$ a $signo(\left\langle w, \psi()\right\rangle - b).$

\begin{enumerate}
    \item Consideramos un conjunto $\mathcal{X}$ y una tarea de aprendizaje (clasificación). Usemos una función $\psi: \mathcal{X} \rightarrow \mathcal{F}.$
    $\mathcal{F} = \R^{n}$ puede ser un espacio
    de Hilbert.
    \item Consideramos una muestra $S = \{(x_{i}, y_{i})\}^{m}_{i = 1} \rightarrow \hat{S} = \{(\psi(x_{i}), y_{i})\}^{m}_{i = 1}.$
    \item Entrenamos con $\hat{S}.$
    \item Predecimos con $(h \cdot \psi)(x) = h(\psi(x)).
    \psi:$ enriquecer nuestra representación con los atributos.
\end{enumerate}

\section{El truco del kernel}

Dada $\psi: \mathcal{X} \rightarrow \mathcal{F}$. Definimos el kernel como

\begin{align}
    \begin{split}
        k(x,x') &= \langle\psi(x), \psi(x')\rangle_{\mathcal{F}}\\
        &= \langle x, x'\rangle_{\psi} \cdot \underset{w}{\text{mín}} L^{H}_{S}(w) + \lambda||w||^{2}_{2}
    \end{split}
    \\ &\Rightarrow \underbrace{\underset{w}{\text{mín}} f(\langle w, \psi(x_{1})\rangle, \langle w, \psi(x_{2})\rangle + R(||w||_{2})}_{\text{(1) Monótona no decreciente}}, \label{eq:1}
\end{align}

\begin{theorem}{Teorema del Representante}

Si $\psi: \mathcal{X} \rightarrow \mathcal{F}$ donde $\mathcal{F}$ es un espacio de Hilbert. Entonces $\exists\ \alpha \in \R^{m}$ tal que $w = \sum^{m}_{i = 1}\alpha_{i}\psi(x_{i})$ y además es una solución a \eqref{eq:1}.

\end{theorem}

\begin{proof}
Sea $w^{*}$ una solución a \eqref{eq:1} y $w^{*} \in \mathcal{F}$

$$w^{*} = \sum^{m}_{i = 1}\alpha_{i}\psi(x_{i}) + u$$

donde $\langle u, \psi(x_{i})\rangle = 0, \forall\ i \in \{1, \dots, m\}$.
Sea $w = w^{*} - u$:

\begin{align}
    \Rightarrow||w^{*}||_{2} = ||w||_{2} + ||u||_{2}\\
    \Rightarrow R(||w||_{2}) \leq R(||w^{*}||_{2})\\
    \langle w, \psi(x_{i})\rangle = \langle w^{*} - u, \psi(x_{i})\rangle = \langle w^{*}, \psi(x_{i})\rangle && \forall\ i \in \{1,\dots, m\}\\
    \Rightarrow f(w) = f(w^{*}
\end{align}

$\Rightarrow w$ es óptimo y además $w = \sum^{m}_{i = 1}\alpha_{i}\psi(x_{i})$ 
\end{proof}

Entonces optimizamos \eqref{eq:1} en términos de $\alpha$.

\begin{align}
    \langle w, \psi(x_{i])}\rangle &= \sum^{m}_{j = 1}\alpha_{j}\langle\psi(x_{j}, \psi(x_{i})\rangle\\ 
    ||w||^{2}_{2} &= \langle w, w\rangle_{\mathcal{F}}\\
    &= \sum^{m}_{j,
     i}\alpha_{j}\alpha_{i}\langle\psi(x_{j}, \psi(x_{i})\rangle\\
     &= \sum^{m}_{j,i}\alpha_{j}\alpha_{i}k(x_{j}, x_{i})\\
     &= \alpha^{T}G\alpha && donde G_{ij} = k(x_{j}, x_{i})\\
\end{align}

\begin{align}
    \Rightarrow\underset{\alpha}{\text{mín}} f(\sum\alpha_{j}k(x_{j}, x_{1}), 
     \dots, \sum\alpha_{i}k(x_{j}, x_{m}) + R(\sqrt{\alpha^{T}G\alpha}) \label{eq:2}\\
     \Rightarrow\underset{\alpha}{\text{mín}} \lambda\alpha^{T}G\alpha + \frac{1}{m} \sum^{m}_{i = 1} \text{máx}\{0, 1 - y_{i}(G\alpha)_{i}\}
\end{align}

\begin{example}{Ejemplos de kernel}
Polinomios de grado k: $k(x, x') = (1 + \langle x, x'\rangle)^{k}; \psi(x) = (1, x,\dots, x^{k}).$\\

Kernel gaussiano: $k(x, x') = exp(\frac{||x - x'||^{2}}{-2\sigma^{2}}; \psi(x) = (\dots, \frac{1}{\sqrt{n!\sigma^{2}}}exp\left(\frac{-x^{2}}{2\sigma^{2}}\right)x^{n},\dots$\\

Hiperbólico: $k(x, x') = tanh(\sigma\langle x, x'\rangle + b)$\\

$k(x, x') = exp\left(\frac{||x - x'||^{2}_{\Sigma}}{-2}\right); ||r||^{2}_{\Sigma} = r^{T}\Sigma^{-1}r, \Sigma = diag(\sigma^{2}_{1},\dots, 
\sigma^{2}_{k})$\\

$k(A, A') = 2^{|A\bigcap A'|}$

\end{example}

\begin{lemma}
El kernel $k: \mathcal{X} \times \mathcal{X} \rightarrow \R$ es válido para definir un producto interior en un 
espacio de Hilbert. $\iff$ Es una función positiva definida. Es decir, $\forall x_{1}, \dots, x_{m}$
la matriz $\psi$ es una matriz positiva definida.
\end{lemma}

\subsection{Propiedades}

Si $k_{1}$ y $k_{2}$ son dos kernel válidos

\begin{enumerate}
    \item $k = k_{1} + k_{2}$
    \item $k = f(x)k_{1}(x, x')f(x')$
    \item $k = exp(k_{1}(x, x'))$
    \item $k = k_{1} \times k_{2}$
    \item $k = k_{1}(x_{a}, x_{a}') \underset{\times}{+} k_{2}(x_{b}, x_{b}'), x = (x_{a}, x_{b})$
\end{enumerate}

%¿Código?

\bibliographystyle{siam} % <<< USE "alpha" BIBLIOGRAPHY STYLE
\bibliography{template} % <<< RENAME TO "lecture_XX"


\end{document}
