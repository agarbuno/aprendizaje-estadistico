\documentclass[11pt,reqno,twoside]{article}
%>>>>>>> RENAME CURRENT FILE TO MATCH LECTURE NUMBER
% E.g., "lecture_01.tex"

%>>>>>>> DO NOT EDIT MACRO FILE

% >>>> DO NOT EDIT THIS FILE
% If you *must* add or change a macro, please email fkoh@caltech.edu

%=================================================
% Basics
%=================================================

\usepackage{fixltx2e} % Makes \( \) equation style robust, among other
                      % things. Must be the first package.


% Makes ligatured fonts searchable and copyable in pdf readers
\usepackage{cmap} % Load before fontenc
\usepackage[spanish]{babel}

% Always include these font encodings in your document
% unless you have a very good reason.
\usepackage[T1]{fontenc}
\usepackage[utf8]{inputenc}

\usepackage{verbatim}

%=====================================
% Look & feel
%=====================================

% Allows for manual space setting
\usepackage{setspace}

%=============
% Fonts
%=============

\usepackage{lmodern} % Improved version of computer modern
\usepackage[scale=0.88]{tgheros} % Helvetica clone for sans serif font


\newcommand\hmmax{2} % Default is 3.
\newcommand\bmmax{2} % Default is 4.

\usepackage{bm} % boldmath must be called after the package
\providecommand{\mathbold}[1]{\bm{#1}}
\usepackage{dsfont}
%=============
% AMS Packages and fonts
%=============
\usepackage{amsmath,amsbsy,amsgen,amscd,amsthm,amsfonts,amssymb}

%=============
% Margins and paper size
%=============
\usepackage[centering,top=1.5in,bottom=1.2in,left=1.4in,right=1.4in]{geometry}

%=============
% Title setup
%=============
\usepackage{titling}
%\usepackage{nopageno}
\setlength{\droptitle}{-7.5em}

\pretitle{\noindent\rule{0.85\linewidth}{0.2mm}\par%
  \begin{raggedright}\LARGE\sffamily}
\posttitle{\par\end{raggedright}%
\noindent%
\rule{0.85\linewidth}{0.5mm}\par}

\preauthor{\noindent\vspace{0.5em}%
  \sffamily\begin{tabular}[t]{ll}}
  \postauthor{\end{tabular}\par\thispagestyle{plain}}

\predate{\noindent%
  \small\sffamily\itshape\begin{tabular}[t]{l}%
    EST-25134, Primavera 2021 \\ %
    Dr.\ Alfredo Garbuno Iñigo \\ %
  }
  \postdate{
  \end{tabular}\par}

%=============
% Section headings
%=============
\usepackage[sf,bf,compact]{titlesec}

%=============
% Tables and lists
%=============
\usepackage{booktabs,longtable,tabu} % Nice tables
\setlength{\tabulinesep}{1mm}
\usepackage[font=small,margin=10pt,labelfont={sf,bf},labelsep={space}]{caption}


\usepackage{enumitem}
\setitemize{itemsep=0pt}
\setenumerate{itemsep=0pt}
\setlist{labelindent=\parindent,%  % Recommended by enumitem package
  font=\sffamily}


%=============
% Hyperlink colors
%=============
\usepackage[usenames,dvipsnames]{xcolor}
\definecolor{dark-gray}{gray}{0.3}
\definecolor{dkgray}{rgb}{.4,.4,.4}
\definecolor{dkblue}{rgb}{0,0,.5}
\definecolor{medblue}{rgb}{0,0,.75}
\definecolor{rust}{rgb}{0.5,0.1,0.1}

\usepackage{url}
\usepackage[colorlinks=true]{hyperref}
\hypersetup{linkcolor=dkblue}
\hypersetup{citecolor=rust}
\hypersetup{urlcolor=rust}
\usepackage[spanish, capitalise]{cleveref}
\usepackage{autonum}
%=============
% Microtype
%=============
\usepackage[final]{microtype}

%=============
% Theorems, etc.
%=============
\newtheoremstyle{myThm} % name
    {\topsep}                    % Space above
    {\topsep}                    % Space below
    {\itshape}                   % Body font
    {}                           % Indent amount
    {\sffamily\bfseries}                   % Theorem head font
    {.}                          % Punctuation after theorem head
    {.5em}                       % Space after theorem head
    {}  % Theorem head spec (can be left empty, meaning ‘normal’)

\newtheoremstyle{myRem} % name
    {\topsep}                    % Space above
    {\topsep}                    % Space below
    {}                   % Body font
    {}                           % Indent amount
    {\sffamily}                   % Theorem head font
    {.}                          % Punctuation after theorem head
    {.5em}                       % Space after theorem head
    {}  % Theorem head spec (can be left empty, meaning ‘normal’)

\newtheoremstyle{myDef} % name
    {\topsep}                    % Space above
    {\topsep}                    % Space below
    {}                   % Body font
    {}                           % Indent amount
    {\sffamily\bfseries}                   % Theorem head font
    {.}                          % Punctuation after theorem head
    {.5em}                       % Space after theorem head
    {}  % Theorem head spec (can be left empty, meaning ‘normal’)

\theoremstyle{myThm}
\newtheorem{theorem}{Teorema}[section]
\newtheorem{lemma}[theorem]{Lema}
\newtheorem{proposition}[theorem]{Proposición}
\newtheorem{corollary}[theorem]{Corolario}
\newtheorem{fact}[theorem]{Dato}

\theoremstyle{myRem}
\newtheorem{remark}[theorem]{Observación}

\theoremstyle{myDef}
\newtheorem{definition}[theorem]{Definición}
\newtheorem{example}[theorem]{Ejemplo}

%=====================
% Header
%=====================
\usepackage{fancyhdr}
\usepackage{nopageno} % Gets rid of page number at the bottom
\fancyhf{} % Clear header style
\renewcommand{\headrulewidth}{0.5pt} % remove the header rule
\pagestyle{fancy}
\fancyhead[LE,RO]{\textsf{\small \thepage}}

\setlength{\headheight}{14pt}
%=====================
% Fix delimiters
%=====================

% Fixes \left and \right spacing issues. See discussion at
% http://tex.stackexchange.com/questions/2607/spacing-around-left-and-right
\let\originalleft\left
\let\originalright\right
\renewcommand{\left}{\mathopen{}\mathclose\bgroup\originalleft}
\renewcommand{\right}{\aftergroup\egroup\originalright}

%=================================================
% Math macros
%=================================================

%=============
% Generalities
%=============
\usepackage{mathtools}
\mathtoolsset{centercolon}  % Makes := typeset correctly for definitions

%%% Equation numbering
%\numberwithin{equation}{section}

%%% Annotations
\newcommand{\notate}[1]{\textcolor{red}{\textbf{[#1]}}}

%==============
% Symbols
%==============
\let\oldphi\phi
\let\oldeps\epsilon
\let\oldemptyset\emptyset
\let\emptyset\varnothing

\renewcommand{\phi}{\varphi}
\renewcommand{\epsilon}{\varepsilon}
\newcommand{\eps}{\varepsilon}
\newcommand{\cl}{\mathrm{cl}}
\newcommand{\wto}{\rightharpoonup}
\newcommand{\wsto}{\overset{\ast}{\rightharpoonup}}
\newcommand{\wwto}{\overset{w}{\to}}
\newcommand{\wwsto}{\overset{w*}{\to}}

%==============
% Constants
%==============

% Set constants upright
\newcommand{\cnst}[1]{\mathrm{#1}}
\newcommand{\econst}{\mathrm{e}}
\newcommand{\rd}{\mathrm{d}}
\newcommand{\dist}{\mathrm{dist}}

\newcommand{\zerovct}{\vct{0}} % Zero vector
\newcommand{\Id}{\mathbf{I}} % Identity matrix
\newcommand{\onemtx}{\bm{1}}
\newcommand{\zeromtx}{\bm{0}}

%==============
% Sets
%==============
\providecommand{\mathbbm}{\mathbb} % In case we don't load bbm

% Reals, complex, naturals, integers, field
\newcommand{\R}{\mathbbm{R}}
\newcommand{\C}{\mathbbm{C}}
\newcommand{\N}{\mathbbm{N}}
\newcommand{\Z}{\mathbbm{Z}}
\newcommand{\F}{\mathbbm{F}}

%==============
% Probability
%==============
\newcommand{\Prob}{\operatorname{\mathbbm{P}}}
\newcommand{\Expect}{\operatorname{\mathbb{E}}}
\newcommand{\D}{\operatorname{\mathcal{D}}}

%==============
% Vectors and matrices
%==============
\newcommand{\vct}[1]{\mathbold{#1}}
\newcommand{\mtx}[1]{\mathbold{#1}}

%=============
% Operators
%=============
\newcommand{\B}{\mathcal{B}}
\newcommand{\op}[1]{\mathbold{#1}}
\renewcommand{\H}{\mathcal{H}}
\newcommand{\X}{\mathcal{X}}
\newcommand{\Y}{\mathcal{Y}}
 % "macro.tex" must be in the same folder

%>>>>>>> IF NEEDED, ADD A NEW FILE WITH YOUR OWN MACROS

% \input{lecture_01_macro.tex} % Name of supplemental macros should match lecture number

%>>>>>>> LECTURE NUMBER AND TITLE
\title{Clase 05:               % UPDATE LECTURE NUMBER
    Convergencia Uniforme}	% UPDATE TITLE
% TIP:  Use "\\" to break the title into more than one line.

%>>>>>>> DATE OF LECTURE
\date{Enero 26, 2021} % Hard-code lecture date. Don't use "\today"

%>>>>>>> NAME OF SCRIBE(S)
\author{%
  Responsable:&
  Emanuel Santiago Payró Costilla  % >>>>> SCRIBE NAME(S)
}

\begin{document}
\maketitle %  LEAVE HERE
% The command above causes the title to be displayed.

%>>>>> DELETE ALL CONTENT UNTIL "\end{document}"
% This is the body of your document.

\begin{itemize}
    \item Recordando lo visto anteriormente, dada una $\mathcal{H}$ finita, en el ERM:
    \begin{enumerate}
        \item Recibimos una muestra $S\sim\mathcal{D}^{m}$.
        \item Evaluamos el $L_{S}$ en cada $h\in\mathcal{H}$
        \item $h_{s}\in\min\limits_{h\in\mathcal{H}}L_{S}(h)$
    \end{enumerate}
    \item Esperamos que $L_{S}(h_{S})$ sea cercana a $L_{\mathcal{D}}(h_{S})$
    \item Necesitamos que todos los riesgos bajo $\mathcal{H}$ sean buenas aproximaciones.
\end{itemize}

\begin{definition}\setlength{\itemsep}{0pt}
    $S$ es una muestra $\epsilon$- representativa con respecto a $Z$, $\mathcal{D}$, $l$ y $\mathcal{H}$ si:
    \begin{center}
        $\forall\hspace{2mm}h\in\mathcal{H},\hspace{6mm}\vert$$L_{S}(h)$\hspace{2mm}-\hspace{2mm}$L_{\mathcal{D}}(h)\vert\leq\epsilon$

    \end{center}

\end{definition}

\begin{lemma}\setlength{\itemsep}{0pt}
    Supongamos que $S$ es un conjunto $\frac{\epsilon}{2}\hspace{1mm}$- representativo (c.r.a. $Z$, $\mathcal{D}$, $l$ y $\mathcal{H}$). Entonces $h_{S}\in\min L_{S}(h)$ satisface:
    \begin{center}
        \begin{minipage}[c]{0.7\linewidth}
        $$L_{\mathcal{D}}(h_{S})\leq \min_{h\in\mathcal{H}}L_{\mathcal{D}}(h)+\epsilon$$
        \end{minipage}
    \end{center}
\end{lemma}

\begin{definition}
    $\mathcal{H}$ tiene la propiedad de convergencia uniforme con respecto a ${Z}$ y $l$ si:
    \begin{center}
        \begin{minipage}[c]{0.9\linewidth}
        $\exists\hspace{1mm} m_{H}^{uc}:(0,1)^{2}\to\hspace{1mm}\N$ tal que $\forall\hspace{1mm}\epsilon,\delta\in(0,1)$ y toda distribución $\mathcal{D}$ sobre $Z$ tenemos que $S$ es una muestra de tamaño m, donde m\hspace{1mm}\geq\hspace{1mm}$m_{\mathcal{H}}^{uc}(\epsilon,\delta)$ entonces con probabilidad mayor o igual a 1 - $\delta$, $S$ es $\epsilon$ - representativo.
        \end{minipage}
    \end{center}
\end{definition}

\begin{corollary}
    Si $\mathcal{H}$ tiene convergencia uniforme con $m_{\mathcal{H}}^{uc}$, entonces la clase puede aprender en el sentido PAC agnóstico, con una complejidad muestral:
    \begin{center}
            $m_{\mathcal{H}}(\epsilon,\delta)\hspace{2mm}\leq\hspace{2mm}m_{\mathcal{H}}^{uc}(\frac{\epsilon}{2},\delta)$
    \end{center}
\end{corollary}

Ahora nos gustaría establecer condiciones para definir cuando una clase de hipótesis es PAC aprendible de manera agnóstica. Para ello, vamos a:
\begin{enumerate}
    \item Acotar la probabilidad de hacer un error generalizable por medio de uniones.
    \item Utilizar un principio de concentración de medida para garantizar la desigualdad.
\end{enumerate}

Para el primer inciso, dados $\epsilon,\delta$ necesitamos encontrar un tamaño de muestra m tal que $\forall\hspace{1mm}\mathcal{D}$ con probabilidad $\geq\hspace{2mm}1-\delta$, la elección de la muestra $S\sim\mathcal{D}^{m}$ garantiza que:\\
\begin{center}
$\forall h\in\mathcal{H}\hspace{6mm}\vert L_{S}(h)-L_{\mathcal{D}}(h)\vert\leq\epsilon$
\end{center}


\begin{minipage}[c]{1.0\linewidth}
    \begin{center}
     \linespread{1.4}\selectfont
        Entonces si tenemos que:
        $\mathcal{D}^{m}\left(\{S:\forall h\in\mathcal{H}, \hspace{2mm}\vert
        L_{S}(h)-L_{\mathcal{D}}(h)\vert\leq\epsilon\right)\geq 1-\delta$ \\
        ó bien $\mathcal{D}^{m}\left(\{S:\exists h\in\mathcal{H}, \hspace{2mm}\vert
        L_{S}(h)-L_{\mathcal{D}}(h)\vert>\epsilon\right)< \delta$\\
        Ahora como: $\{S:\exists h\in\mathcal{H}, \hspace{2mm}\vert L_{S}(h)-L_{\mathcal{D}}(h)\vert >\epsilon\}=\bigcup\limits_{h\in\mathcal{H}}\{S:\hspace{1mm}\vert L_{S}(h)-L_{\mathcal{D}}(h)\vert >\epsilon\}$\\
$\Rightarrow\mathcal{D}^{m}\left(\{S:\exists h\in\mathcal{H}, \hspace{2mm}\vert
        L_{S}(h)-L_{\mathcal{D}}(h)\vert>\epsilon\right)\leq\hspace{1mm}\sum\limits_{h\in\mathcal{H}}\mathcal{D}^{m}\left(\{S:\hspace{1mm}\vert L_{S}(h)-L_{\mathcal{D}}(h)\vert >\epsilon\}\right)$
    \end{center}

\end{minipage}

\begin{lemma}
$\textbf{(Desigualdad de Hoeffding)}$ Si $\theta_{1}, ..., \theta_{m}\hspace{1mm}$ son independientes e idénticamente distribuidas, tal que $\forall i\hspace{2mm} \Expect(\theta i)=\mu\hspace{2mm}$y $\hspace{2mm}\mathbb{P}(a\leq\theta\leq b)=1$, entonces:
    \begin{center}
        \begin{minipage}[c]{0.7\linewidth}
            $\mathbb{P}(\vert\Bar{\theta}_{m}-\mu\vert>\epsilon)\leq\hspace{1mm}2\exp\left(-\displaystyle\frac{2m\epsilon^{2}}{(b-a)^{2}}\right)$ donde $\Bar{\theta}_{m}=\frac{1}{m}\sum\limits_{i=1}^{m}\theta_{i}$
        \end{minipage}
    \end{center}
\begin{center}
    \linespread{1.4}\selectfont
    $\theta_{i}=l(h,z_{i}); z_{i}\overset{\text{iid}}{\sim}\mathcal{D};\theta_{i}\hspace{2mm}iid.$\\
    $\Bar{\theta}_{m}=L_{S}(\theta)\hspace{10mm}\mu=L_{\mathcal{D}}(h)$\\
    $l(h,)\in[0,1]\hspace{6mm}\Rightarrow\hspace{6mm}\theta_{i}\in[0,1]$\\
    $\mathcal{D}^{m}(\{S:\vert L_{S}(h)-L_{D}(h)\vert>\epsilon\})=\mathbb{P}(\vert\Bar{\theta}_{m}-\mu\vert>\epsilon)\leq\hspace{2mm} 2\exp(-2m\epsilon^{2})$\\
    $\Rightarrow\mathcal{D}^{m}(\{S:\exists h\in\mathcal{H}, \vert L_{S}(h)-L_{D}(h)\vert>\epsilon\})\leq \sum\limits_{h\in\mathcal{H}}2\exp(-2m\epsilon^{2})$\\
    $=2\vert\mathcal{H}\vert\exp(-2m\epsilon^{2})$\\
    $\Rightarrow m\geq\displaystyle\frac{\log(\frac{2\vert\mathcal{H}\vert}{\delta})}{2\epsilon^{2}}$
\end{center}

\end{lemma}

\begin{corollary}
    Sea $\mathcal{H}$ una clase finita de hipótesis, $\mathcal{Z}$ el dominio y l: $\mathcal{H}\times\mathcal{Z}\to[0,1]$ una función de pérdida. Entonces $\mathcal{H}$ posee la propiedad de convergencia uniforme con complejidad muestral:\\
    \begin{center}
            $m_{\mathcal{H}}^{uc}(\epsilon,\delta)\leq\Bigg\lceil\displaystyle\frac{\log(\frac{2\vert\mathcal{H}\vert}{\delta})}{2\epsilon^{2}}\Bigg\rceil$
    \end{center}
    Además, el algoritmo ERM utilizando una complejidad muestral:\\
    \begin{center}
        $m_{\mathcal{H}}(\epsilon,\delta)\hspace{2mm}\leq\hspace{2mm}m_{\mathcal{H}}^{uc}(\frac{\epsilon}{2},\delta)$
    \end{center}
    Nos garantiza que $\mathcal{H}$ es aprendible como PAC agnóstico.

\end{corollary}


    \textbf{Resumen:} Si tenemos convergencia uniforme para $\mathcal{H}$ entonces $L_{S}$ estará cercano a $L_{D}$\\
    \begin{center}
        Si CU + ERM $\Rightarrow$ PAC agnóstico
    \end{center}

%>>>>>> END OF YOUR CONTENT
\bibliographystyle{siam} % <<< USE "alpha" BIBLIOGRAPHY STYLE
\bibliography{template} % <<< RENAME TO "lecture_XX"


\end{document}
